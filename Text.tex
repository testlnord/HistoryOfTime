\section{Аннотация}
Мы живем в очень интересное время. Время, когда у практически у каждого есть мобильный телефон в кармане, персональный компьютер на рабочем столе, телевизор на кухне и практически неограниченное «облачное» хранилище информации. Время, когда со всех сторон нас окружают машины. А что же превращает эти машины из груды железа в вещи, к которым мы так привыкли? \\
Эту важную роль на себя берет программное обеспечение. ПО играет роль души, которая делает разнообразные устройства так близки нам. \\
Для того что бы сделать программное обеспечение наиболее удачным, нужен хороший инструмент. Инструментом для создания ПО является язык программирования. \\
На сайте The Language List\footnote{Киннерсли Б. ``Language List'' официальный сайт Канзаского университета URL:http://people.ku.edu/~nkinners/LangList/Extras/langlist.htm дата обращения: 18.09.2012}  
сейчас представлено около 2500 языков, но сколько из них реально используется и почему только несколько их них получили широкое распространение в среде программистов? \\
В нашей работе мы поставили цель, выяснить какие факторы привели к появлению и распространению современных наиболее популярных и интересных нам языков программирования. \\
Для достижения цели нужно ответить на вопросы: как развивались языки программирования, какие области программного обеспечения они охватывали, какие парадигмы программирования поддерживали и какие факторы повлияли на популяризацию данного языка программирования. Для ответов на данный вопрос нужно решить следующие задачи. Первое, какие языки имеют наибольший успех среди программистов. Второе, составить хронологическую модель развития языков. И третье рассмотреть каждый язык в отдельности, выделить его основные характеристики и определить факторы, из-за которых он получил свою  репутацию.